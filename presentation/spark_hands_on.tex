\documentclass[slidetop,9pt,utf8]{beamer}

\usepackage{CJKutf8}
\usepackage[utf8]{inputenc}
\usepackage[english]{babel}
\usepackage{ifpdf}
\usepackage{textcomp}
\usepackage{color}
\usepackage{xcolor}
\usepackage{amsmath}
\usepackage{graphics}
\usepackage{array}
\usepackage{graphicx}
\usepackage{colortbl}
\usepackage{pinyin}
\usepackage{algorithm,algorithmic}
\usepackage{hyperref}
\usepackage{listings}
\usepackage{caption}

\usetheme{JuanLesPins}
\useoutertheme{infolines}
%\usecolortheme{default}
%\usetheme{Boadilla}
\setbeamerfont{structure}{size*={11}{11}}

\DeclareCaptionFont{white}{\color{white}}
\DeclareCaptionFormat{listing}{\colorbox[cmyk]{0.43, 0.35, 0.35,0.01}{\parbox{\textwidth}{\hspace{15pt}#1#2#3}}}
\captionsetup[lstlisting]{format=listing,labelfont=white,textfont=white, singlelinecheck=false, margin=0pt, font={bf,footnotesize}}

\lstset{
         basicstyle=\footnotesize\ttfamily,
         numberstyle=\tiny,          
         numbersep=5pt,              
         tabsize=2,                  
         extendedchars=true,         
         breaklines=true,            
         keywordstyle=\color{red},
         showspaces=false,           
         showtabs=false,             
         xleftmargin=17pt,
         xrightmargin=17pt,
         framexleftmargin=5pt,
         framexrightmargin=5pt,
         framexbottommargin=4pt,
         showstringspaces=false      
 }

\lstdefinestyle{terminal}
{
    backgroundcolor=\color{black},
    basicstyle=\scriptsize\color{white}\ttfamily
}

\lstdefinestyle{code}
{
    frame=b,
    numbers=left
    %numberstyle=\tiny\color{mygray}
}

% "define" Scala
\lstdefinelanguage{scala}{
  morekeywords={abstract,case,catch,class,def,%
    do,else,extends,false,final,finally,%
    for,if,implicit,import,match,mixin,%
    new,null,object,override,package,%
    private,protected,requires,return,sealed,%
    super,this,throw,trait,true,try,%
    type,val,var,while,with,yield},
  otherkeywords={=>,<-,<\%,<:,>:,\#,@},
  sensitive=true,
  morecomment=[l]{//},
  morecomment=[n]{/*}{*/},
  morestring=[b]",
  morestring=[b]',
  morestring=[b]"""
}

\begin{document}
\begin{CJK}{UTF8}{gbsn}


%--- Page de titre ---
\title{Spark Hands On}
%\subtitle{Creation of a multi-features Named Entity Disambiguation System}
\author{Olivier GIRARDOT, Vincent DOBA}
\institute{Paris Scala User Group}
\date{May 13, 2015}

\frame{\titlepage}

%--- Sommaire ---
\begin{frame}
  \frametitle{Content}
  \tableofcontents[hideallsubsections]
\end{frame}

\section{Prerequisite}

%--- Prerequisite for the machine ---
\begin{frame}
  \frametitle{Machine Prerequisite}

  \begin{block}{You should have a machine with}
    \begin{itemize}
      \item JDK 6+ (this hands on was developed using Oracle JDK 8) with Scala 2.10+
      \item SBT 13+
      \item Git
      \item An IDE with SBT/Scala support (for instance IDEA IntelliJ with SBT and Scala plugins)
      \item Downloaded Spark 1.3.1 prebuilt for Hadoop 2.6 from \href{http://www.apache.org/dyn/closer.cgi/spark/spark-1.3.1/spark-1.3.1-bin-hadoop2.6.tgz}{official Apache Spark site}
    \end{itemize}
  \end{block}

  \begin{block}{You should}
    \begin{itemize}
      \item Know Scala basis
      \item Be at ease with Scala's iterables transformations (map, flatmap, fold)
      \item Be at ease with higher order functions
    \end{itemize}
  \end{block}

\end{frame}

\begin{frame}[fragile]
  \frametitle{First Spark Run}

  \begin{block}{Perform following commands}
    \begin{lstlisting}[language=bash, style=terminal]
git clone https://github.com/vincentdoba/spark-hands-on.git
cd spark-hands-on
sbt "run-main psug.hands.on.prerequisite.WordCount README.md" 
    \end{lstlisting}
  \end{block}

  \begin{block}{You should obtain}
    \begin{lstlisting}[language=bash, style=terminal]
[info] Running psug.hands.on.prerequisite.WordCount README.md
[error] 15/05/08 17:37:36 WARN NativeCodeLoader: Unable to load native-hadoop library for your platform... using builtin-java classes where applicable
[info] (,48)
[info] (*,24)
[info] (####,16)
[info] (of,14)
[info] (the,11)
[info] (###,10)
[info] (Exercise,8)
[info] (Description,8)
[info] (Notions,8)
[info] (a,8)
[success] Total time: 25 s, completed 8 mai 2015 17:37:39
    \end{lstlisting}
  \end{block}

\end{frame}

\begin{frame}[fragile]
  \frametitle{Words Count Script}

  \lstinputlisting[language=scala, style=code, label=WordCount,caption=/src/main/scala/psug/hands/on/prerequisite/WordCount.scala]{../src/main/scala/psug/hands/on/prerequisite/WordCount.scala}

\end{frame}

\section{Exercise 1 : Spark basis (SparkContext, RDD, Transformation, Action)}
\section{Exercise 2 : Key/Values RDD, File Loading}
\section{Exercise 3 : Spark SQL Context, JSON Loading, DataFrames}
\section{Exercise 4 : Transformations/Actions on DataFrames, Join}
\section{Exercise 5 : Transformations/Actions on DataFrames}
\section{Exercise 6 : Saving JSON file}
\section{Exercise 7 : Machine Learning}


\end{CJK}
\end{document}
